%%%%%%%%%%%%%%%%%%%%%%%%%%%%%%%%%%%%%%%%%%%%%%%%%%%%%%%%%%%%%%%%%%%%%%%%%%%%%%%%
%2345678901234567890123456789012345678901234567890123456789012345678901234567890
%        1         2         3         4         5         6         7         8

\documentclass[letterpaper, 10 pt, conference]{ieeeconf}  % Comment this line out if you need a4paper

%\documentclass[a4paper, 10pt, conference]{ieeeconf}      % Use this line for a4 paper

\IEEEoverridecommandlockouts                              % This command is only needed if 
                                                          % you want to use the \thanks command

\overrideIEEEmargins                                      % Needed to meet printer requirements.

% The following packages can be found on http:\\www.ctan.org
\usepackage{graphics} % for pdf, bitmapped graphics files
\usepackage{graphicx}
\usepackage{amsmath,amssymb,latexsym,float,epsfig,subfigure}
\usepackage{url}
% \usepackage[export]{adjustbox}
%\usepackage{epsfig} % for postscript graphics files
%\usepackage{mathptmx} % assumes new font selection scheme installed
%\usepackage{times} % assumes new font selection scheme installed
\usepackage{amsmath} % assumes amsmath package installed
\usepackage{amssymb}  % assumes amsmath package installed
\usepackage{lipsum}
\usepackage[export]{adjustbox}
\usepackage[normalem]{ulem} % underline
\usepackage{wrapfig}
%\usepackage[leftcaption]{sidecap}
\usepackage{multirow}
\usepackage{color}
\title{\LARGE \bf
	Human-in-the-Loop Optimization of Shared Autonomy\\in Assistive Robotics
}


\author{Deepak Gopinath$^{1,3}$, Siddarth Jain$^{2,3}$ and Brenna D. Argall$^{1\text{-}4}$% <-this % stops a space
	%\thanks{*This work was not supported by any organization}% <-this % stops a space
	\thanks{$^{1}$Department of Mechanical Engineering,
		Northwestern University, Evanston, IL, 60208, USA
		%        {\tt\small deepakedakkattilgopinath2015@u.northwestern.edu}}%
	}
	\thanks{$^{2}$
		Department of Electrical Engineering \& Computer Science,
		Northwestern University, Evanston, IL, 60208, USA
		%        {\tt\small sidd@u.northwestern.edu}}%
	} 
	%\thanks{ {\tt\small deepakedakkattilgopinath2015@u.northwestern.edu}}%
	%\thanks{ {\tt\small sidd@u.northwestern.edu}}%
	%\thanks{ {\tt\small brenna.argall@northwestern.edu}}%
	\thanks{$^{3}$
		%	Department of Electrical Engineering and Computer Science, 
		%	Northwestern University, Evanston, IL, 60208, USA
		Rehabilitation Institute of Chicago, Chicago IL, 60211 USA
	}
	\thanks{$^{4}$
			Department of Physical Medicine and Rehabilitation, 
			Northwestern University, Chicago IL, 60611 USA	
	}
	\thanks{\hspace{-0.5cm} {\tt\small deepakedakkattilgopinath2015@u.northwestern.edu}}%
	\thanks{ \hspace{-0.5cm}{\tt\small sidd@u.northwestern.edu}}%
	\thanks{ \hspace{-0.5cm}{\tt\small brenna.argall@northwestern.edu}}%
}

\begin{document}
	
	
	
	\maketitle
	\thispagestyle{empty}
	\pagestyle{empty}
	
	
	%%%%%%%%%%%%%%%%%%%%%%%%%%%%%%%%%%%%%%%%%%%%%%%%%%%%%%%%%%%%%%%%%%%%%%%%%%%%%%%%
	\begin{abstract}
		In this paper, we propose a mathematical framework which formalizes user-driven customization of shared autonomy in assistive robotics as a nonlinear optimization problem. 
		%Users of assistive devices differ in
		%their physical abilities and desired amount of assistance which
		%motivates the customization of how control is shared between the human
		%and robot.
		Our insight is to allow the \textit{end-user}, rather than relying on standard optimization techniques, to perform the optimization procedure, thereby allowing us to leave the exact nature of the cost function indeterminate.
		We ground our formalism with an interactive
		optimization procedure that customizes
		control sharing using an assistive robotic arm. We also present
		a pilot study that explores interactive optimization with
		end-users.
		This study was performed with 17 subjects (4 with spinal cord injury, 13 without injury). Results
		show all subjects were able to converge to an assistance paradigm, suggesting the existence of optimal solutions. Notably, the amount of assistance was not always
		optimized for task performance. Instead, some subjects
		favored retaining more control during the execution over better task
		performance.
		%the user
		%  satisfaction gained from accomplishing the task on their own over
		%  better performance. 
		The study supports the case for user-driven customization and provides
		guidance for its continued development and study.
	\end{abstract}
	
	
	%%%%%%%%%%%%%%%%%%%%%%%%%%%%%%%%%%%%%%%%%%%%%%%%%%%%%%%%%%%%%%%%%%%%%%%%%%%%%%%%
	\section{INTRODUCTION} \label{Intro}
	
	For people with severe motor impairments as a result of spinal cord or
	brain injuries, assistive and rehabilitation machines such as
	assistive robotic arms, upper or lower limb prostheses and powered
	wheelchairs are crucial for reducing their dependence on caretakers
	and increasing the ability to perform activities of daily life.
	% in an easier way. 
	However, for many, the control of such devices remains a
	challenge---for example, due to their physical impairments or limitations of the control interfaces. Limited interfaces issue control signals that are low-dimensional, discrete and operate in modes which correspond to different parts of the control space that must be switched between. The introduction of
	partial autonomy to these devices---in which the control is shared
	between the human and robotics autonomy---aims to help reduce the cognitive and physical burden on the user. 
	
	The reduced bandwidth of the control signals generated by motor-impaired users makes them more reliant on the interaction with the autonomy, and also less adaptable and more vulnerable to any arbitrariness present in the system---for example, the choice of control interfaces and mappings, or the exact specification of how control is shared between the user and the autonomy. A thorough analysis of user performance and the differences in performance between uninjured and motor-impaired subjects calls for a rich mathematical framework that can capture the various facets of the shared control system, like the complex dynamics of the human-robot interaction. \textit{Furthermore}, the exact formulation used to describe the human-robot interaction will determine (or limit) the relevant and valid questions that can be 
	asked and how the analysis of performance metrics will be performed. To accomplish this, we introduce a mathematical formalism in which the customization procedure is formulated as a nonlinear optimization problem over system parameters. 
	
	Since users differ in their
	physical abilities and desired amount of assistance, \textit{customization} of the amount of assistance is critical for the adoption of assistive shared-control systems.
	Predefined assistance levels 
	can provide good starting points but may not remain optimal for the user in the long term. For example, the subject's abilities will likely be changing---either degrading (e.g.~due to degenerative disease) or improving (e.g.~due to successful rehabilitation). As a result, the need for assistance may increase or decrease. One way to accomplish customization is to tune the system parameters which will bring about a change in the human-robot interaction and the final behavior. The aim is to optimize the human-robot interaction during task performance. A straightforward choice of optimality criterion is to consider task-related performance metrics such as minimizing the time taken and energy expended. Such metrics however may not capture user-related metrics like comfort, independence or satisfaction.
	
	Our insight is that if we entrust the task of customization to the users, they likely will tune the system in such a way that the optimal interaction---according to their personal optimality criterion---will emerge. Moreover, the user-driven customization of assistance may be user-dependent in addition to being task-dependent. 
	%Our proposed formalism furthermore allows for time varying interaction. 
	
	To ground our formalism, we present a first implementation, in which the reasoning between the user control
	and the robot policy is a function of confidence in the inference of human intent, with tunable parameters. 
	Our interactive user-driven customization system maps verbal cues from the human to adjustments
	in these parameter values. Results from an exploratory pilot study also are presented.
	
	In Section~\ref{rw} we present an overview of the relevant research in
	the field of shared control systems in assistive
	technology. Section~\ref{pf} overviews of the general algorithm and
	system design used in this study. The system implementation is
	described in Section~\ref{si} and Section~\ref{PSM} provides
	% describes the system implementation. In Section~\ref{usm}, we present 
	an overview of the
	%tasks performed by the subjects during the study,
	user study methods, tasks and metrics used.
	% for evaluation. 
	In Section~\ref{RES} we present the results from our pilot study and discussion followed by conclusions in Section~\ref{CON}.
	
	\section{RELATED WORK} \label{rw}
	
	%The introduction of robotics autonomy to assistive devices can offload some control burden from users and enable easier operation. The most common methods to share control between the user and the autonomous system involve user selection of the higher level goal and autonomy generation of the lower-level control, control partitioning schemes  and blending the user controls and the autonomy commands. In the domain of robotic wheelchair research, the higher-level goals typically are navigation goals [] while control partitioning may place the control of the speed with the user and the heading with the autonomy, for example []. Control blending paradigms are often employed for behaviors like obstacle avoidance []. Control sharing in case of robotic arms most commonly involves user specification of a target (such as an object) and the robot autonomously assists in reaching the target~\cite{tsui2011want, jain2010assistive}. Approaches that partition the control space for example may place the control of end-effector position in $z$ with the human and of $x$, $y$ with the autonomy []. Recently, approaches in which the user's commands are blended with the autonomy commands to perform manipulation tasks have gained interest.
	
	The introduction of robotics autonomy to assistive devices
	can offload some control burden from users to enable
	easier operation. While full autonomy is an option, more common are systems that share control with the human user---for reasons of both robustness~\cite{05icorr-volosyak} and user preference~\cite{01smc-kim}.
	
	The most common methods to share control
	between the user and autonomous system include (a) the user
	selects the higher-level goal and the autonomy generates
	the lower-level control, (b) control partitioning schemes and (c)
	blending the user controls and the autonomy commands.
	In the domain of robotic wheelchair research, the higher-level goals typically are navigation goals~\cite{05jrrd-simpson}, while control
	partitioning for example places the control of speed with the user
	and heading with the autonomy~\cite{NAVCHAIR}. Control
	blending paradigms often are employed for behaviors like
	obstacle avoidance~\cite{12smc-carlson}. Control sharing in case of robotic arms
	most commonly involves user-specification of a target (such
	as an object)~\cite{jain2010assistive} or pose correction~\cite{04ar-bien}, and the robot autonomously generates the motion commands. 
	Approaches that partition the control space
	may, for example, place the control of end-effector position in
	$z$ with the human and in $x,y$ with the autonomy~\cite{05uais-driessen}.
	Control blending is less straightforward---because the user rarely is able to issue a control signal with high enough dimensionality to cover all control dimensions of the robot (e.g. 6D)---but recently is gaining interest.
	
	Moreover, there are approaches which study specifically the customization of how this control blending happens~\cite{ li2011dynamic}. The amount of control blending often is determined using an arbitration function that is based, for example, on the autonomy's confidence in its prediction of the user's goal~\cite{dragan2012formalizing}. Our work similarly employs an arbitration function to dictate the amount of control sharing. 
	
	% in which robotics autonomy can help with behaviors such as obstacle avoidance and navigation~\cite{trieu2008shared, yanco2000shared, simpson2005smart}. Control sharing becomes more difficult in case of assistive robotic arms, as robotic arms have higher degree-of-freedom than their
	%control interfaces. One approach is to have the user specify the target to the robot while keeping the control of the robot autonomous. Although more convenient, such schemes result in loss of control for users and autonomous behaviors moreover are prone to failures~\cite{volosyak2005rehabilitation}. Alternatively, some works target to make  the teleoperation of the robot easier by mapping the user's input to a subset of the control space~\cite{Mulling-RSS-15, sjain, driessen} and keep the user in the control loop. 
	%
	%There exist approaches that studies customizable autonomy levels and blending of user's commands with the autonomy control commands~\cite{Dragan_2012_7224, li2011dynamic}.  Additionally, the amount of control blending can be determined using a arbitration function of robot's confidence in its prediction of user's target~\cite{dragan2012formalizing}. In our work, the arbitration function is of the same form as in~\cite{dragan2012formalizing} and we target to customize the assistance in a easier and interactive way, via direct interaction with the user.
	
	Optimization techniques have been adopted to generate different strategies for control sharing; for example, formulating the problem as a POMDP and inferring a distribution over goals~\cite{javdani2015shared}, using pseudo-navigation functions for collaborative control~\cite{fernandez2015towards} or concatenating energy-optimal motion primitives to create optimal trajectories~\cite{lawitzky2013trajectory}. Although these approaches result in improved task performance (completion time, control effort), the assistance schemes are mixed in terms of user acceptance. In particular, there are instances of assistance resulting in higher user dissatisfaction~\cite{javdani2015shared}, and users preferring to be in control and more cautious~\cite{lawitzky2013trajectory}. In other studies users find the assistance at times to be uncooperative and tolerate a loss of control only for a significant improvement in performance~\cite{you2012assisted}.
	%Researchers have adopted different strategies for shared autonomy in assistive teleoperation; for example, using reactive potential field and sample based planning to generate robot policies incorporating dynamic constraints (e.g.~collision avoidance) and. ~\cite{you2012assisted}. 
	
	In an attempt to construct more realistic cost functions, others inspired by design research incorporate a measure of ``discomfort" into the optimization cost function~\cite{gulati2009framework}. However, the specific form of the cost function is domain dependent and is not generalizable to other assistive devices such as robotic arms. 
	
	Despite an improvement in task performance, none of the above cost function formulations were able to guarantee high user satisfaction (with the exception of domain-specific discomfort~\cite{gulati2009framework}). The need for higher user satisfaction is crucial for the acceptance of robot autonomy by the end-users in the assistive domain. This gap motivates our approach to engage the end-user in the optimization procedure.
	
	\section{PROPOSED FRAMEWORK} \label{pf}
	
	Principles from optimal control theory have been successfully used to account for different aspects of human motor control such as  arm trajectory formation, posture control and locomotion \cite{ uno1989formation, flash1985coordination,todorov2004optimality}.
	The underlying motivation in using optimal control theory is that biological systems have evolved to produce motor commands which will optimize motor behavior with respect to the task at hand \cite{todorov2004optimality}. When a human operates an assistive robot to replace his/her lost motor function, the extension of this reasoning is that the optimizing principles are operating over control commands to the robot effector rather than motor commands to the human muscles.
	We frame our formalism within the language of optimal control theory not only because of this biological parallel, but also because it will allow for the analysis of the effects of the various design decisions in and components of a shared control system in a thorough and rigorous manner. 
	%However, motor behavior studies have primarily focused on reaching and grasping tasks performed by humans with no motor impairments. For human subjects with motor impairments due to spinal cord or brain injuries, the above mentioned cost function need not account for the task behavior. 
	
	\subsection{Formalism}
	Let $\boldsymbol{x}(t)$ denote the state of the system at time $t$. 
	%where $\boldsymbol{x}(t)$ is the 6-d pose of the end effector of the robot. 
	Let $\boldsymbol{\theta}(t)$ be the set of tunable parameters that will affect the amount of control shared between the human and the robot. 
	The other control inputs to the system are $\boldsymbol{u}_h(t)$ and $\boldsymbol{u}_r(t)$, the control commands generated respectively by the user and autonomous robot policy at time $t$. 
	
	The control signal from the robot autonomy is generated by a function $f_{r}(\cdot) \in \mathcal{F}_{r}$, 
	\begin{equation}
		\boldsymbol{u}_r(t) \leftarrow f_{r}(\boldsymbol{x}(t))
	\end{equation}
	where $\mathcal{F}_{r}$ is the set of all control behaviors corresponding to different tasks.
	
	We assume that the control command $\boldsymbol{u}_h(t)$ is generated by a function of  $f_{h}(\cdot) \in \mathcal{F}_{h}$,
	\begin{equation}
		\boldsymbol{u}_h(t) \leftarrow f_{h}(\boldsymbol{x}(t))
	\end{equation}
	where $\mathcal{F}_{h}$ is the set of user behaviors corresponding to different tasks. $f_{h}(\cdot)$ is simply a symbolic representation of the mapping function that generates $\boldsymbol{u}_{h}(t)$ and is completely unknown to the autonomous system.
	The sole dependence of $\boldsymbol{u}_h(t)$ on $\boldsymbol{x}(t)$ is an approximation because there may be a large number of unobserved variables (e.g.~fatigue or personal satisfaction) affecting the user's control. 
	
	The shared control system makes use of function $\boldsymbol{\beta}(\cdot)$, parameterized by $\boldsymbol{\theta}$
	\begin{equation}
		\boldsymbol{u}(t) \leftarrow \boldsymbol{\beta}_{\theta}(\boldsymbol{u}_h(t), \boldsymbol{u}_r(t))
	\end{equation}
	which arbitrates between the control commands from the user and the robot policy to produce control command $\boldsymbol{u}(t)$ executed by the robot.
	
	A key insight in our formulation is that, for a time--varying function $\boldsymbol{\beta}(\cdot)$, the parameters themselves can be functions of time and therefore may be interpreted as control signals. Then the dynamics of the system can be written as
	\begin{equation}
		\boldsymbol{\dot{x}}(t) \leftarrow \boldsymbol{a}(\boldsymbol{x}(t), \boldsymbol{\theta}(t), \boldsymbol{u}_h(t), \boldsymbol{u}_r(t), t)
	\end{equation}
	where $\boldsymbol{a}(\cdot)$ is in general a nonlinear, time-varying function. Note that in this formulation the parameters $\boldsymbol{\theta}(t)$ are treated in the \textit{same way} as the other control signals.
	The problem of finding the set of parameters $\boldsymbol{\theta}(t)$ that will generate the optimal human-robot interaction and task performance (as determined by a cost function) thus may be formulated as an optimal control problem.
	
	Optimal control models assume the existence of some kind of cost function being optimized during task performance.\footnote{For example, arm trajectories generated by uninjured humans during reaching tasks are reproduced using cost functions composed of torque generated at the joints~\cite{uno1989formation} or jerk of the end effector (hand)~\cite{flash1985coordination}.} In general, the cost function $\boldsymbol{J}$ can be written as,  
	\begin{equation}
		\boldsymbol{J} \leftarrow \boldsymbol{h}(\boldsymbol{x}(t_f), t_f) + \int\limits_{t_0}^{t_f} \boldsymbol{k}(\boldsymbol{x}(t), \boldsymbol{u}(t), t)dt
	\end{equation}
where the first term corresponds to a terminal cost (e.g~proximity of the end pose to the target pose) and the second term corresponds to a measure of internal cost (e.g~energy expended, completion time, etc). The true cost function however likely is more complex, and could include additional factors such as user satisfaction. The state boundary conditions are given by 
	$
	\boldsymbol{x}(t_0)	= x_0 
	$ and
	$
	\boldsymbol{x}(t_f) = x_f
	$,
	where $t_0$ and $t_f$ are the times at which the robot is in the initial state $x_0$ and final state $x_f$. 
	The parameter constraints are
	$
	\boldsymbol{\theta}_{min} \leq \boldsymbol{\theta}(t) \leq \boldsymbol{\theta}_{max}
	$.
	
	The elements of the framework $f_{r}(\cdot)$, $f_{h}(\cdot)$, $\beta(\cdot)$ and 
	$a(\cdot)$ are system-specific, and different choices of these functions will have drastically different impact on task performance and user satisfaction. Moreover, the impact is anticipated to be all the greater on motor-impaired subjects.
	\begin{figure}
		\begin{center}
			\includegraphics[width = 1\hsize]{./finalfigures/Figure1.eps}
			\vspace{-0.5cm}
			\caption{System design. Core components include command arbitration
				and the interactive optimization of how this arbitration happens.}
			\label{SD}
		\end{center}
	\end{figure}
	\subsection{Optimization}
	Typically optimization is performed over all control signals that are inputs to the system. In our system, however, the control commands from the human and the robot ($ \boldsymbol{u}_h(t), \boldsymbol{u}_r(t)$) are treated as given quantities, and the goal rather is to optimize the interaction parameters $\boldsymbol{\theta}(t)$. 
%	Therefore, optimization is performed only with respect to a subspace ($\boldsymbol{\theta}(t)$) of the entire control space. 
	
	In this work, we furthermore make no attempt to determine the exact nature of the cost function $\boldsymbol{J}$. There might be a myriad of unmeasurable factors influencing the cost function, and determining the exact mathematical form for the cost function likely is an intractable problem. Making any kind of approximation to simplify the cost function in turn will affect the robustness and efficacy of the assistive system. Since we do not want to reduce the assistive capabilities of our system, and we have a human in the loop, our insight is that the optimization task can be performed by the user him/herself, instead of adopting standard nonlinear optimization algorithms. Thus there is no need to concretely define $J$ and the user tunes the parameters $\boldsymbol{\theta}(t)$ until the desired behavior is achieved. 
	
	In this user-driven customization system, the overall effect of parameter tuning is that of changing the assistance offered by the robot. The specific optimization procedure is described in detail in Section \ref{UDOAP}.	
	
	\section{SYSTEM IMPLEMENTATION} \label{si}
	
	
	Our system implementation is overviewed in Figure~\ref{SD}. 
	The key components of this system include the command arbitration paradigm (Sec. \ref{CA}),
	the user-driven optimization procedure (Sec. \ref{UDOAP}) and the estimation of human intent (Sec. \ref{ECV}).
	Also provided are details of how the human control signals are acquired (Sec. \ref{CIM}) and the robot autonomy
	commands are generated (Sec. \ref{DRP}). 
	
	\subsection{Command Arbitration} \label{CA}
	%\vspace{0.2cm}
	
	In our implementation, the function $\beta(\cdot)$ that reasons between the robot and human control signals
	is a linear blending function,
	\begin{equation}
		\beta_{\boldsymbol{\theta}}(\boldsymbol{u}_{\textit{h}}(t),\boldsymbol{u}_{\textit{r}}(t)) \triangleq (1-\alpha_{\boldsymbol{\theta}}) \cdot \boldsymbol{u}_{\textit{h}}(t) +  \alpha_{\boldsymbol{\theta}} \cdot \boldsymbol{u}_{\textit{r}}(t) %\newline
		\label{eq:blend}
	\end{equation}
	where $\alpha_{\boldsymbol{\theta}} \in [0,1]$ is itself a function parameterized by $\boldsymbol{\theta}$.\footnote{The time index $t$ is dropped from $\boldsymbol{\theta}(t)$ for brevity in notation.}
	%Here the \textit{confidence} ($c(t)$) refers to the autonomous system's confidence in its inference of human intent at time $t$ and will be described in detail in Sec. \ref{ECV}. 
	Note that $\alpha_{\boldsymbol{\theta}}=0$ corresponds to full teleoperation,
	and $\alpha_{\boldsymbol{\theta}}=1$ to full autonomy.
	
	The majority of \textit{arbitration functions} $\alpha_{\boldsymbol{\theta}}$ can be reduced to the functional
	form pictured in Figure~\ref{PAF} \cite{dragan2012formalizing}, characterized by a set of three parameters $\{\theta_{1}, \theta_{2}, \theta_{3}\}$ and independent variable $c(t)$. The parameter set determines:
	\begin{itemize}
		\item $\theta_{1}$: The minimum value of $c(t)$ above which control blending is performed. 
		\vspace{0.1cm}
		\item $\theta_{2}$: The value of $c(t)$ above which the blending parameter is maximum and constant. 
		\vspace{0.1cm}
		\item $\theta_{3}$: The maximum value of $\alpha$ for any value of $c(t)$. 
		\vspace{0.1cm}
	\end{itemize}
	
	\begin{wrapfigure}[12]{R}{0.27\textwidth}
		\begin{center}
%			\vspace{-.6cm}
			\includegraphics[width=0.27\textwidth]{./finalfigures/Figure2.eps}
		\end{center}
		\vspace{-.45cm}
		\caption{A prototypical arbitration function, parameterized by {\footnotesize $\boldsymbol{\theta} = \{\theta_{1}, \theta_{2}, \theta_{3}\}$}.}
		\label{PAF}
	\end{wrapfigure}
	Note that $\theta_{3}$\;=\;$0$ corresponds to constant teleoperation (irrespective of the value of $c(t)$). The relationship between $c(t)$ and $\alpha_{\boldsymbol{\theta}}$ is linear between $\theta_{1}$ and $\theta_{2}$, and the slope of this linear relation determines how aggressively the robot assumes control.
	The parameter bounds are such that $\forall i,~ \theta_{i}(t) \in [0,1]$ and $\theta_{1} \leq \theta_{2}$.
	The independent variable $c(t)$ in our implementation is discussed in Sec \ref{ECV}. 
	
	In our pilot study, the parameters were tuned only between tasks and were unchanged during task execution, and so $\boldsymbol{\theta}_{i}(t) = \boldsymbol{\theta}_{i}(t_{0}),\; \forall t \in [t_{0}, t_{f}]$.
	The arbitrated signal $\boldsymbol{u}(t)$ was the velocity of the end-effector in Cartesian space, converted to joint-space velocities via inverse kinematics.
	%\vspace{-0.22cm}
	\subsection{User--Driven Optimization of the Arbitration Parameters} \label{UDOAP}
	
	In this first exploration of our interactive optimization
	procedure, verbal commands from the human subject are translated to changes in $\boldsymbol{\theta}$ by the system operator. The interactive optimization procedure is currently being formalized and automated, as informed by this pilot data. 
	
	A change in assistance level can be achieved by modulating one or more of the $\theta_{i} \in \boldsymbol{\theta}$, according to
	$\theta_{i} = \theta_{i} \pm \delta\theta_{i}$.
	In our implementation, at initialization $\delta\theta_{i} =
	0.1$.
	% and $c_{max} \geq c_{min}$
	The value of $\delta\theta_{i}$ is adaptive, and is halved if a request to increase assistance is immediately followed by a request to decrease and vice versa (in order to avoid oscillatory behavior). 
	
	\begin{table}[t]
		\centering
		\begin{tabular}{|l|l|l|}
			\hline
			%  \multicolumn{3}{|c|}{\textbf{Control Mappings}} \\
			{Verbal Cue} & {Parameters Changed} & {Amount of change}\\
			\hline
			``More'' & $\theta_{3}\uparrow$, $\theta_{2}\downarrow$, $\theta_{1}\downarrow$ & $\delta\theta \leftarrow \delta\theta$
			\\ \hline
			``Less'' & $\theta_{3}\downarrow$, $\theta_{2}\uparrow$, $\theta_{1}\downarrow$ &  $\delta\theta \leftarrow \delta\theta$ \\ \hline
			``Little More'' & $\theta_{3}\uparrow$, $\theta_{2}\downarrow$, $\theta_{1}\uparrow$ & $\delta\theta \leftarrow \frac{1}{2} \delta\theta$ \\ [5pt] 	 \hline
			``Little Less'' & $\theta_{3}\downarrow$, $\theta_{2}\uparrow$, $\theta_{1}\;(no\;change)$  & $\delta\theta \leftarrow \frac{1}{2} \delta\theta$ \\  \hline
			%		``More Aggressive'' & $(\theta_{2} - \theta_{1}) \downarrow$, $\theta_{3}\uparrow$ & $d\theta \leftarrow d\theta$ \\ \hline
			%		``More Timid'' & $(\theta_{2} - \theta_{1}) \uparrow$, $\theta_{3}\downarrow$ & $d\theta \leftarrow d\theta$ \\ \hline
		\end{tabular}
		\label{tbl:maptable}
		\caption{Mappings from verbal cues to parameters changed}
		\vspace{-.4cm}
		($\uparrow$ indicates a positive $\delta\theta$ and $\downarrow$ denotes a negative $\delta\theta$)
		\vspace{-0.2cm}
	\end{table}
	Table I provides a few example mappings between common
	verbal cues, the parameters changed and the values of $\delta\theta$. We chose to modulate more than one parameter at a time as it helped to make the change in assistance level more perceivable to the user.
	
	\subsection{Estimation of Intent} \label{ECV} 
	In our implementation, the independent variable $c(t)$ of the arbitration function is the autonomous system's confidence in its inference of the intent (goal) of the human. The confidence $c(t)$ is computed at each execution step
	that the human provides a control signal, i.e. $\boldsymbol{u}_{h}(t) \ne
	\emptyset$.
	In our implementation, $c(t)$ is computed as
	\begin{equation}
		%G_{A} = \bf{w_{1}({\vec{v}}_{user}.{\vec{v}}_{A}) + w_{2}(e^{-d})}
		c(t) \triangleq \boldsymbol{w}_{1}(\boldsymbol{u}_\textit{h}(t) \cdot \boldsymbol{u}_\textit{r}(t)) + \boldsymbol{w}_{2}(\boldsymbol{e}^{-d})
		\label{eq:conf}
	\end{equation}
	where $d$ is the Euclidean distance between the end effector and an
	inferred target location at time $t$, and $c(t) \in [0,1]$.
	The first term in (\ref{eq:conf}) 
	provides a measure of
	agreement between the user-generated commands and robot-generated commands.\footnote{Commands $\boldsymbol{u}_{h}(t)$ and $\boldsymbol{u}_{r}(t)$ are first smoothed using a moving average filter ($0.6s$), so that small command changes do not affect the confidence measure drastically.}
	%% This measure of alignment is
	%% important for computing the confidence measure because, the robot
	%% policy was originally derived from human demonstrations and takes into
	%% account other factors in the environment such as obstacles. 
	The second term encodes the nearness to the target.
	%, which is another good indicator of the user's intended goal.
	%The exponential ensures that the distance measure is in the range 0 to 1. 
	Parameters $\boldsymbol{w_{1}}$ and $\boldsymbol{w_{2}}$ are
	task-specific weights.
	
	At each execution step this confidence measure is computed for all
	candidate goals in the scene, $g \in \mathcal{G}$, resulting in a
	%. Thus, we have a
	distribution of confidences $c_{g}(t) \in \mathcal{C}$
	%, \forall g \in \mathcal{G}$ 
	over the candidate goals.\footnote{In the pilot study the candidate goals are objects placed at predefined positions in front of the robot. Our system also is able to autonomously perceive object positions and use these as candidate goals.} To compute these confidences,
	each control behavior $f_g$ generates a command (where $f_g$ aims to
	reach candidate goal $g$) which is used in the calculation of
	$c_g(t)$ according to (\ref{eq:conf}).
	The command associated with the target that has the highest computed confidence is selected.
	\vspace*{-0.15cm}
	\subsection{Control Interface and Mapping} \label{CIM}
	The human control command $\boldsymbol{u}_h(t)$ is captured via a teleoperation interface, that consists of a 3-axis joystick operated under two different mapping paradigms (Table~\ref{tbl:modes}). The joystick signals are mapped to the translational and rotational velocities of the end-effector in Cartesian space.
	\begin{figure*}[t]
		\centering
		\includegraphics[width = 1\hsize]{./finalfigures/Figure3_eps.eps}
		\caption{Study tasks  performed by SCI participant. \textit{Left to right:} Simple Reaching (R), Reaching for Grasping (RfG), Reaching for Scooping (RfS).}
		\label{TOne}
	\end{figure*}
	The first paradigm uses only two of the three axes (no twist)---because many
	end-users lack the hand function to perform twisting---and accordingly
	defines four 2D modes to cover the six control dimensions of the robot
	arm. We refer to this as the \textit{2D mapping paradigm}. The second uses all three of the joystick axes under two 3D modes and is referred to as the \textit{3D mapping paradigm}. 
	
	
	
	\begin{table}
		\centering
		\begin{tabular}{|l|l|l|}
			\hline
			\multicolumn{3}{|c|}{Control Mappings} \\
			\hline
			\textbf{Mode} & \textbf{3D} & \textbf{2D}\\
			\hline
			1 & $v_{x},v_{y},v_{z}$ & $v_{x}, v_{y}$ \\ \hline
			2 & $\omega_{x}, \omega_{y}, \omega_{z}$   & $v_{x}, v_{z}$ \\ \hline
			3 &           ---                   &  $\omega_{x}, \omega_{y}$ \\ \hline
			4 &           ---                   & $\omega_{z}$ \\ \hline
		\end{tabular}
		\caption{Operational paradigms for the teleoperation interface} 
		\label{tbl:modes}
		\vspace{-.5cm}
	\end{table}
	
	\subsection{Derivation of the Autonomy Policy} \label{DRP}
	The robot control command $\boldsymbol{u}_{r}(t)$ is generated from an autonomous control policy. 
	While any number of techniques may be employed to
	derive the behavior functions in $\mathcal{F}_{r}$, there are some
	limitations on the form that $f_r(\cdot)$ should take. Attempting to return the robot to the
	pre-planned path (as many planners do) does not make sense in
	shared-control systems where the sources of deviations are human commands---this likely would be unwelcome to the
	user. Instead replanning would need to happen fast enough not to stall the task
	execution. We therefore advocate the use of real-time control
	policies which are defined in
	all parts of the state space.
	
	Our current implementation favors dynamical systems formulations.
	The autonomous robot policies are learned from human demonstrations using an approach known as \textit{Stable Estimator of Dynamical Systems (SEDS)} \cite{khansari2011learning}. In SEDS, the target poses are modeled as attractors of a dynamical system. For each task, a set of $N$ demonstrations are collected by
	kinesthetically moving the robot. Demonstrations consist of pairs of joint angles $\boldsymbol{x}(t) \in \mathbb{R}^{6}$ and joint velocities $\boldsymbol{\dot{x}}(t) \in \mathbb{R}^{6}$. The SEDS algorithm learns the parameters of a time-independent dynamical system which models the joint velocities as a function of joint angles, and so
	\begin{equation}
		\boldsymbol{\dot{x}}(t) \leftarrow f_{r}(\boldsymbol{x}(t))
	\end{equation}
	and $\boldsymbol{u}_{r}(t) \triangleq K(\boldsymbol{\dot{x}}(t))$ where $K(\cdot)$ is the forward kinematics of the robot arm (since human teleoperation and control blending both happen in the  end-effector Cartesian space). The dynamical system ensures that the policy exists everywhere in the workspace, and that the robot trajectories follow the general contour of the task demonstrations. 
	
	\section{PILOT STUDY METHODS} \label{PSM}
	
	The experiments were
	performed using the MICO robotic arm (Kinova Robotics, Canada) which
	is specifically designed for assistive purposes. The system was
	implemented using the Robot Operating System (ROS) and model learning
	was performed using MATLAB. The maximum end effector translational velocity along any axis was capped at 20 cm/s. 
	\subsection{Task Descriptions}
	Three tasks were developed for our pilot study (Fig.~\ref{TOne}).
	
	\vspace{0.1cm}
	\noindent \uline{\textit{Simple Reaching (R):}}
	The user teleoperates the robotic arm to reach a single object (coffee carafe) placed in front of the robotic arm. The purpose of this task is to get the user accustomed to the control interface and to the different assistance levels provided by the system. At the end of the task, the assistance level that the user preferred was noted.
	
	\vspace{0.1cm}
	\noindent \uline{\textit{Reaching for Grasping (RfG):}}
	The user teleoperates the robotic arm to reach one of two objects on
	the table with a pose suitable for grasping, as the robot arm
	provides assistance. There is a near object (mug) and a far object
	(box), each of which requires a different orientation of the gripper
	for grasping (side and top, respectively) and accordingly also
	different approach trajectories during reaching.  
	
	\vspace{0.1cm}
	\noindent \uline{\textit{Reaching for Scooping (RfS):}}
	The user teleoperates the robotic arm to reach for one of two objects
	on the table with a pose suitable for a scooping
	motion, as the robot arm provides assistance. There is a near object
	and a far object (both bowls), each of which requires a different 
	approach trajectory.
	For this task, the end effector of the robotic arm is fitted with a
	spoon which must be inserted into the bowl.
	
	\subsection{Study Protocol and Metrics}
	\begin{figure*}[t]
		\centering
		%	\includegraphics[scale = 0.3]{./figures/RevisedPlot12.jpg}
		\includegraphics[width=1.18\hsize, center]{./finalfigures/Figure4_1.eps}
		\vspace{-0.7cm}
		\caption{Task completion time (top row) and number of mode switches (bottom row) for uninjured vs. SCI subjects (first column), Task 1 vs. Task 2 for uninjured subjects only (second column), Task 1 vs. Task 2 for SCI subjects only (third column).}
		\label{PlotOne}
	\end{figure*}
	\vspace{0.1cm}
	\noindent{\uline{\textit{Subjects:}}}  For this exploratory
	study 17 subjects were recruited---13 uninjured control subjects (mean
	age: $26 \pm 4$, 8 males and 5 females) and 4 spinal cord injury (SCI) subjects (mean age: $35 \pm 14$, all males, C3-C5 injury
	levels). Seven of the uninjured subjects (5 males, 2 females) and
	3 of the SCI subjects used the \textit{3D} interface paradigm, and the remaining
	subjects used the \textit{2D} paradigm.  All participants gave their
	informed, signed consent to participate in this experiment, which was
	approved by Northwestern University's Institutional Review Board.
	
	\vspace{0.1cm}
	\noindent{\uline{\textit{Protocol:}}} Each user performed all three
	tasks. The purpose of the practice task (R) was to get the user
	accustomed to the control interface and assistance system. Data was
	then collected on the remaining two tasks (RfG, RfS). The order of presentation
	for the RfG and RfS tasks was randomized and balanced across
	subjects, to avoid ordering effects.
	
	Before the RfG and RfS trials, the user was first asked to operate the system in full teleoperation mode (\textit{tel}) and also under three predefined assistance levels (\textit{min, mid} and \textit{max}). After this phase, the subject was given the option to customize the assistance level. Changes in assistance levels were communicated verbally to the system operator resulting in the parameter changes outlined in Table I. The user then tested the customized assistance level by executing the task. This customization procedure was repeated until the user was satisfied and lasted on average 10 and a maximum of 15 minutes, resulting in assistance level \textit{custom}. Data collection began only after this customization process was completed. Three trials were collected for \textit{min, max} and \textit{custom} assistance levels.\footnote{For one SCI participant one less trial was recorded for \textit{min} assistance level during the first task due to a clerical error.} A typical session lasted approximately 1-1.5 hours. 
	For the first (non-practice) task, the baseline from which customization began was the \textit{mid} level assistance, with level \textit{custom} being the result after customization. For the second task, customization began at this level \textit{custom} from the first task as the baseline, with the option to further customize resulting in level \textit{custom} for the second task.
	
	\noindent{\uline{\textit{Metrics:}}} A number of objective metrics evaluated this study. \textit{Task Completion time} was the amount of time spent accomplishing a task. \textit{Mode Switches} refers to the number of times the subject switched between the various modes of the control interface (Table \ref{tbl:modes}). Mode switches additionally is an indirect measure of the effort put forth by the user. At the
	end of the study, subjective data was gathered via a brief
	questionnaire. Users were given
	statements about the assistance system to rate on a 7-point Likert scale (1 is low, 7 is high), according to
	their agreement. The questions primarily concerned the utility value of the assistance system (\textit{U1}), the system's accuracy in goal perception (\textit{CA1}) and its understanding of what the user is trying to accomplish (\textit{CA2}), and the contribution from the user (\textit{CO1}) and the system (\textit{CO2}) in task accomplishment.
	
	\section{RESULTS} \label{RES}
	Here we report the results of our pilot study.\footnote{The video of the study can be found at \url{http://argallab.smpp.northwestern.edu/index.php/publications/}} An improvement in task performance with customization is demonstrated, and a number of other interesting observations are noted. Task performance metrics for different assistance levels (denoted by \textit{min}, \textit{max} and \textit{custom} in the plots) and teleoperation (\textit{tel)} are analyzed across different subject groups, tasks and control interfaces. Note that the \textit{custom} assistance level always lies in between (or is equal to) \textit{min} and \textit{max}. 
	Statistical significance is determined by Welch t-tests for Figures~\ref{PlotOne}-\ref{PlotTwo} and two sided Wilcoxon Rank-Sum Test for Figure~\ref{RCAP}, where (***) indicates $p<0.001$, (**) $p<0.01$ and (*) $p<0.05$.
	\subsection{Observations across Uninjured and SCI subjects}
	\noindent{\uline{\textit{Insight into Cost Function:}}}
%	Figure \ref{PlotOne} represents the task completion time (top) and mode switches (bottom) and for each task for uninjured (column 2) and SCI (column 3) subjects. 
%	For the \textit{custom} assistance type, task completion times are comparable to those of the \textit{max} assistance type.
%	However, the number of mode switches for \textit{custom} is \textit{greater} than those of \textit{max} (though this effect is not statistically significant). 
%	Thus, it is not uniformly the case that mode switches are minimized from low to high assistance. 
	In this study, 17 subjects performed 34 rounds of customization in total. For 7 customization rounds the mean \textit{custom} task completion time was greater (by at least one standard error) than that of \textit{max}. Similarly, the number of mode switches for \textit{custom} was greater than that of \textit{max} for 14 customization rounds. This indicates that subjects are not always optimizing for standard performance metrics---because there does exist a parametrization (\textit{max}) which was known to the subjects and performs better with respect to these metrics.
	%\vspace*{-0.09cm}
	This  provides insight that the true cost function that the user is optimizing likely is more complex than a simple time-optimal or minimum-effort cost function. 
	
	\vspace{0.1cm}
	\noindent{\uline{\textit{Task Performance:}}} 
	In Figure \ref{PlotOne} (first column), the difference between uninjured and SCI subjects' task completion times drops steadily from \textit{tel} to \textit{custom} assistance levels. The t-tests revealed that while the difference between uninjured and SCI was statistically significant for \textit{tel} ($p$ = $5.1e$-$4$), \textit{min} ($p$ = $6.5e$-$5$) and \textit{max} ($p$ = $0.027$), this difference disappeared with the \textit{custom} ($p$ = $0.096$) assistance level. That is, with customized assistance, the performance of SCI subjects was \textit{statistically equivalent} to that of uninjured subjects. The variance in the data also \textit{decreases} with customized assistance, showing the performance to become more consistent. 
	
	Interestingly, for mode switches there was no statistical difference between uninjured and SCI subject data for any of the assistance levels. 
	This suggests that the number of mode switches is primarily determined by the nature of the task and control interface, and not the state of injury. However, SCI subjects do take more time than uninjured subjects to perform the same number of mode switches. 
	\vspace{-0.1cm}
	\subsection{Observations across Tasks}
	Figure \ref{PlotOne} (second and third columns) shows how task completion times and number of mode switches change between the first and second task for uninjured and SCI subjects. A statistically significant difference in performance only is observed for \textit{custom} assistance, for both groups. 
%	For uninjured subjects, there was no change in performance over trials, and neither between different assistance levels for either metric. The same was true for SCI subjects with the \textit{exception} of the \textit{custom} level. 
	Interestingly, SCI subjects show an \textit{improvement} in task completion times ($p$ = $7.8e$-$3$) and mode switches ($p$ = $8.9e$-$3$) between the first and second tasks, whereas uninjured subjects exhibit a performance \textit{decrease}. These changes in performance can be explained by the changes in assistance amount that result from the between-task customization (discussed further in Section~\ref{RCPC}).
%	This indicates that subjects were able to better understand and evaluate their own preferences and capabilities in between tasks and thereby make the necessary changes to the assistance level. 
%	 There is also significant improvement in task completion times ($p$ = $3.2e$-$3$ for Task 1 and $p$ = $8.2e$-$4$ for Task 2) and mode switches ($p$ = $1.6e$-$4$ for Task 1 and $p$ = $4.9e$-$5$ for Task 2) for levels \textit{tel} and \textit{custom}.
	\begin{figure}[t]
		\centering
		%	\includegraphics[width = 0.5\textwidth]{./figures/RevisedPlot23.jpg}
		\includegraphics[width = 1.2\hsize, center]{./finalfigures/Figure5_new4.eps}
		\vspace{-0.5cm}
		\caption{Left Column: Task completion time (top) and number of mode switches (bottom) for the 2D vs. 3D interfaces. Right Column: Within-interface assistance comparison for the 2D (top) and 3D (bottom) interfaces.}
		\label{PlotTwo}
	\end{figure}
	\subsection{Observations across Control Interfaces}
	Figure \ref{PlotTwo} (first column) shows the task completion times and mode switches for subjects using the \textit{2D} and \textit{3D} interfaces. Different operational modes do not seem to have an effect on task completion times, as both groups are statistically equivalent---\textit{despite} the fact that for mode switches the difference between the \textit{2D} and \textit{3D} interfaces is significant. The second column of Figure \ref{PlotTwo} shows a \textit{within-interface} performance comparison between \textit{tel} and the different assistance levels. For all levels assistance significantly helped in reducing the number of mode switches during task execution.
	
	The comparable task completion times may be explained by the fact that easier control compensates for time lost during mode switches. That is, due to the greater number of mode switches required for the \textit{2D} interface compared to the \textit{3D} interface, more time is taken performing mode switches. However, the number of dimensions simultaneously controlled is less for the \textit{2D} interface compared to the \textit{3D} interface, which makes the control easier.
	
	\subsection{Relative Change in Parameters during Customization} \label{RCPC}
	Figure~\ref{RCAP} shows the change in amount of assistance (parameter values) during customization for uninjured and SCI subjects. While SCI subjects on average increased the amount of assistance ($p$\;=\;$0.020$) during the second phase of customization, uninjured subjects chose to reduce the amount of assistance ($p$ = $0.006$). By contrast, there are no noticeable changes in the amount of assistance when using the \textit{2D} versus \textit{3D} interface. Injury thus seems to be the primary factor in how subjects choose to change the customized assistance level, and the mapping paradigm seems to have little effect. It furthermore is interesting that uninjured subjects chose to reduce assistance in spite of an associated decrease in task performance.
	%\begin{wrapfigure}{l}{0.2\textwidth}
	%	\begin{center}
	%		\includegraphics[width=0.2\textwidth]{./figures/RevisedPlot3.jpg}
	%	\end{center}
	%	 	\vspace{-1.9cm}
	%	\caption{Relative change in assistance parameters during customization: Uninjured vs. SCI (top) and 2D vs. 3D (bottom).}
	%	\label{RCAP}
	%\end{wrapfigure}
	\begin{figure}[t]
		\centering
		\includegraphics[scale=0.5]{./finalfigures/Figure6_new4.eps}
		\vspace{-0.6cm}
		\caption{Relative change in assistance parameters during customization for Uninjured vs. SCI subjects (left) and 2D vs. 3D interfaces (right).}
		\label{RCAP}
	\end{figure}
	\subsection{User Survey}
	%On an average users had quite positive responses to the different category of questions. When asked about the utility value of the system the responses had a mean of .....  The users also thought that the assistance system was capable of doing the right things as on an avergae it perceived the goals and intent quite successfully. Despite this users felt that they were equally responsible for the task accomplishment maybe indicating that they were not prepared to relinquish control altogether.
	Users rated (Fig.~\ref{US}) the utility value of the assistance system fairly high (mean = $5.9\pm0.8$) indicating that in general having assistance was favored. The users also thought that the system was able to perceive goals accurately (mean = $5.1\pm1.8$) and the inability to estimate human intent was fairly low (mean = $3.1\pm1.1$). The users also felt that they played an important part in accomplishing the task (mean = $5.5\pm1.0$), almost comparable to the contribution from assistance (mean = $5.1\pm1.6$), maybe indicating that they were not prepared to relinquish control altogether.
	\begin{figure}[t]
		%	\centering
		\includegraphics[scale = 0.42, center]{./finalfigures/Figure7.eps}
		\vspace{-0.5cm}
		\caption{User responses on perceived utility, contribution and capability.}
		\label{US}
	\end{figure}
	\vspace{-0.08cm}
	\subsection{Discussion}
	From our pilot study we saw that compared to pre-defined assistance levels, customization improves task performance and helps to reduce performance differences between uninjured and SCI subjects. Post-experiment surveys also revealed that the users found the customized assistance paradigm to be useful. These results establish a need for customization of assistance levels. Therefore, our next step will be to explore multiple possibilities for building effective and intuitive customization mechanisms (e.g.~physical interfaces operated by the user) which will suit individual requirements and preferences, and to evaluate on a larger end-user population. The results also show that the true cost function that is being optimized is more complex than a simple time-optimal or minimum-effort cost function, indicating the need to investigate the exact specification of the true cost function that is being optimized by the human in a shared control system. A more comprehensive user survey will also be administered in which the formalized customization procedure will be evaluated thoroughly.
	%\addtolength{\textheight}{-0.1cm}  
	\section{CONCLUSIONS}\label{CON}
	In this work, we formalized human robot interaction in shared autonomy within the framework of optimal control theory. Furthermore we introduced a system for user-driven customization as a constrained nonlinear optimization problem within this framework. Unlike standard optimization problems in which the form of the cost function is predetermined in this work no such assumptions were made. Instead, the end user was allowed to directly perform the optimization procedure. The aim is that this will lead to higher user satisfaction, which is crucial for the acceptance of novel technologies in the assistive domain. An interactive user-driven customization system was developed to ground the formalism and the results from the pilot study were presented. Results showed that all subjects were able to converge to a optimal assistance paradigm, and an improvement in task performance with customization also was demonstrated.
	\section*{ACKNOWLEDGMENTS}\label{ACK}
	Many thanks to Jessica Presperin Pedersen, OTR/L,
	ATP/SMS, for recruiting the SCI subjects, and to Samuel
	Schlesinger for assistance during the subject studies. Research reported in this publication was supported by the
	NIBIB \& NICHD
	under award number R01EB019335. The content is solely
	the responsibility of the authors and does not necessarily
	represent the official views of the NIH.
	
	%
	%For the data-recording trials, the user was first asked to operate the
	%system in full teleoperation mode and also under three predefined
	%assistance levels (\textit{min, mid, max}). After this phase, the
	%subject was given the opportunity to further customize the assistance level
	%in an interactive and iterative manner. This customization phase lasts
	%a maximum of 10-15 minutes. Once the customization phase was over, four trials per
	%assistance level (\textit{min, mid, max} and \textit{custom}) were
	%recorded. The presentation order of the various assistance levels was
	%random and balanced.
	%%, to avoid ordering effects.
	%%Verbal commands from the subject such as ``more assistance", ``less assistance", ``somewhere between middle and maximum" etc determine the exact nature of customization. For example, an increase (decrease) in assistance level is accomplished by increasing (decreasing) $\alpha_{max}$ and reducing (increasing) the difference between $c_{thresh}$ and $c_{max}$. 
	%A typical session lasted approximately 1-1.5 hours.
	%%(healthy-SCI) hours.
	%
	%For the first data-recording task, the baseline from which
	%customization began was the \textit{mid} level assistance. The
	%result after customization is referred to as \textit{CustomA}. For
	%the second data-recording task, customization began using
	%\textit{CustomA} as the baseline, and the result is \textit{CustomB}. The intent here was to investigate
	%to what extent customization is task-specific.
	%
	%
	%
	%
	%\vspace{0.2cm}
	%\noindent{\textit{Study Metrics:}}  A number of objective
	%metrics evaluated this study. \textit{Task completion time} is the
	%amount of time spend on accomplishing a task. \textit{Mode switches}
	%refer to the number of times the subject switched between the various
	%modes of the control interface (Tbl.~\ref{tbl:modes}). The above
	%mentioned metrics are indirect measures of the effort put forth by the
	%user. At the end of the study, subjective data was gathered via a
	%brief questionnaire. To explore subject opinions, users were given
	%statements about the assistance system, and asked to rate each one of
	%them on a 7-point Likert scale, according to their agreement (where one
	%indicates strong disagreement, seven strong agreement).
	%%. A value of one
	%%indicates strong disagreement, a value of four denotes neutral,
	%%whereas a value of seven is strong agreement. The different categories
	%%in the questionnaire and one example question from each of the
	%%categories are as follows:
	%\begin{itemize}
	%\item \textit{Utility:} I can achieve the task more easily with the robot's assistance than without it.
	%\item \textit{Contribution:} Assistance contributed to task success.% of the task.
	%\item \textit{Trust:} I trust the robot to do the right thing.
	%% at the right time.
	%%\item Capability - The robot perceives accurately what my goals are.
	%%\item Traits - The robot is committed to the task.
	%\end{itemize}
	%The questionnaire also had questions relating to self-reported demographics, such as their familiarity with new technology.
	%\begin{figure}[t]
	%	\centering
	%	\includegraphics[width=1\hsize]{./figures/TimeCombined_crop}
	%	\vspace{-.7cm}
	%	\caption{Task completion times for all tasks at all assistance levels. Mean over trials with standard error bars.}
	%	\label{fig:time}
	%\end{figure}
	%
	%\begin{figure}[t]
	%	\centering
	%	\vspace{-0.1cm}
	%	\includegraphics[width=.97\hsize]{./figures/ModeSwitchCombined_crop}
	%	\vspace{-.2cm}
	%	\caption{Average number of mode switches for all tasks at all assistance levels. Mean over trials with standard error bars.}
	%	\label{fig:modes}
	%\end{figure}
	%
	%\section{RESULTS} \label{res}
	%
	%Here we report the results of our pilot study. A general
	%improvement in performance with customization is demonstrated, and a
	%number of interesting observations are noted.
	%
	%\begin{figure}[t]
	%\centering
	%\includegraphics[width=.49\hsize]{./figures/TimeDiffCatTelCustomA_crop}
	%\includegraphics[width=.49\hsize]{./figures/MSDiffCatTelCustomA_crop}\\
	%%% \includegraphics[width=.65\hsize]{./figures/TimeDiffCatTelCustomA}\\
	%%% \vspace{-.9cm}
	%%% \includegraphics[width=.65\hsize]{./figures/MSDiffCatTelCustomA}\\
	%\vspace{-.2cm}
	%\caption{A sampling of task completion time (top) and 
	%  mode switching (bottom) data, broken out by subject population. Task RfG-Far, control sharing levels \textit{teleop} (no assistance) and \textit{CustomA} (first task customization). Mean over trials with standard error bars.}
	%\label{fig:subj_pop}
	%\end{figure}
	%
	%\begin{figure*}[t]
	%%\begin{center}
	%%\centering
	%%\includegraphics[scale = 0.4]{./figures/ArbFuncChangeNumEBNew}
	%\begin{tabular}{ccc}
	%\multirow{5}{*}{\includegraphics[width=0.3\hsize]{./figures/ArbFuncChangeNumEbNew_crop}} & & \\
	%& \multirow{4}{*}{\includegraphics[width=0.36\hsize]{./figures/NoTrialsCustomize_crop}} &\\
	%&& \includegraphics[width=0.3\hsize]{./figures/UserOpinionStdError_crop}\\
	%&&\\
	%&&\\
	%\end{tabular}
	%\vspace{.8cm}
	%\caption{\textit{Left:} Relative change in arbitration function
	%  parameters during customization. Mean over
	%  tasks. \textit{Middle:} Number of trials to perform customization,
	%  during each customization phase. Mean over tasks. \textit{Right:}
	%  User questionnaire responses. Mean over subjects. All plots report
	%  standard error bars. }
	%%\vspace{-1.0em}
	%\label{fig:cust}
	%%\end{center}
	%\end{figure*}
	%
	%
	%\subsection{Task Performance  with Customization}
	%
	%Figure~\ref{fig:time} reports the task completion time for all
	%assistance levels, subjects and tasks, using both the 3D (top) and 2D
	%(bottom) interfaces. We observe broadly that task completion times
	%improve with assistance. 
	%
	%There is however no conclusive differences between interfaces types
	%with respect task completion time. However, we do see a distinct
	%difference in the number of mode switches, which increase with the
	%number of operational modes as we would expect (Fig.~\ref{fig:modes}).
	%
	%
	%%For both of these performance metrics, we cannot say anything
	%%conclusive about which assistance level, beyond not-min
	%
	%
	%\subsection{Task Performance between Subject Populations}
	%
	%Figure~\ref{fig:subj_pop} teases apart the subject data by population,
	%to see if there are any performance differences. We do observe a
	%difference under teleoperation---namely, that task completion time is
	%longer for SCI subjects. 
	%
	%A promising observation however is that task completion times
	%normalize with assistanc---across subject populations, as well
	%as interface modality. Moreover, mode switches virtually disappear
	%with customized assistance.
	%
	%An interesting observation is that the number of mode switches when
	%operating the 2D interface is higher for uninjured subjects. This
	%perhaps speaks to a lack of familiarity in operating under such
	%control constraints.
	%
	%
	%
	%
	%
	%
	%
	%\subsection{Customization Trends and User Opinion}
	%
	%
	%Figure~\ref{fig:cust} (left, center) shows the relative change in the
	%arbitration function parameters through customization, and the number
	%of trials taken to perform this customization. We indeed see that
	%there is still a noticeable level of customization happening in Phase
	%2, suggesting that arbitration function customization likely is
	%task-specific. What is interesting however is that the number of
	%trials taken to perform this customization decreases in Phase
	%2---suggesting that perhaps users are becoming more adept at
	%performing the customization procedure. We also see that users are largely 
	%
	%Figure~\ref{fig:cust} (right) also reports on user opinion of the
	%assistance system and its utility to them. The responses are on the average quite positive.
	%
	%
	%% ------ %
	%
	%%% %Plots regarding time, mode switches. Aggregates maybe? People preferences and how they varied. Discussion based on observations and conversations with the subjects during the experiment. How SCI prefers low to mid as their preferred customized levels. More time and mode switches is not an issue or concern. Prefes having control. Question of trust with the robot policy is a issue. Can possibly be improved by collecting demostration data itself from the user. When the complexity of task goes up then more assistance is sought after. 
	%%% %% In order to evaluate the potential of this proposed approach in which user driven customization of assistance level is possible, we looked at both the objective and subjective study metrics as outlined in the previous section. In this section we first present the objective data and then the subjective data gathered from the questionnaire. 
	%
	%%% Figure~\ref{time} and Figure~\ref{mode} show the time taken and the number of mode switches for all the tasks for 3D control mapping and 2D control mappings and all assistance levels averaged over all subjects. Similarly, 
	%%% From Figure~\ref{time}, it can be seen that task completion times are lower for customized assistance levels and are usually comparable to middle level assistance level. The number of mode switches is also less when the system operates at the customized assistance level. Furthermore, the number of mode switches is in general greater when the 2D control interface mapping is used. 
	%
	%%% \begin{figure}[h]
	%%% \begin{center}
	%%% \includegraphics[width = 0.5\textwidth]{./figures/TimeCombined}
	%%% \caption{Task completion times for all tasks at all assistance levels - 3D and 2D control interfaces.}
	%%% \label{time}
	%%% \end{center}
	%%% \end{figure}
	%
	%
	%%% The data from the questionnaire is presented in Figure~\ref{QData}. This data is a good measure of the users' perception of the system, that is, how well they accepted the robot's assistance policy and the comfort and trust they have in the system. All the questions were answered assuming that the system was using the user-customized assistance level. 
	%
	%%% For the question regarding the utility of the system, all users weakly agreed that they will be able to perform the tasks more easily without the robot's assistance. However, this data is in direct contraction with the objective metrics collected on task performance. 
	%%% %\begin{figure}[H]
	%%% %\begin{center}
	%%% %\includegraphics[width = 0.5\textwidth]{./figures/TMSReaching3d}
	%%% %\caption{Time Taken and Number of Mode Switches for Reaching Task - 3D control}
	%%% %\label{TMSR}
	%%% %\end{center}
	%%% %\end{figure}
	%%% %\begin{figure}[H]
	%%% %\begin{center}
	%%% %\includegraphics[scale = 0.4]{./figures/TMSScooping3d}
	%%% %\caption{Time Taken and Number of Mode Switches for Scooping Task - 3D Control}
	%%% %\label{TMSS}
	%%% %\end{center}
	%%% %\end{figure}
	%%% %
	%%% %\begin{figure}[H]
	%%% %\begin{center}
	%%% %\includegraphics[width = 0.5\textwidth]{./figures/TMSReaching2d2}
	%%% %\caption{Time Taken and Number of Mode Switches for Reaching Task - 2D Control}
	%%% %\label{TMSR2d}
	%%% %\end{center}
	%%% %\end{figure}
	%%% %\begin{figure}[H]
	%%% %\centering
	%%% %\includegraphics[scale= 0.4]{./figures/TMSScooping2d2}
	%%% %\caption{Time Taken and Number of Mode Switches for Scooping Task - 2D Control}
	%%% %\label{TMSS2d}
	%%% %\end{center}
	%%% %\end{figure}
	%%% It can be clearly seen that during pure teleoperation of the robot (no robot assistance), the task completion times and the number of mode switches are the highest for both control interface mappings and tasks. 
	%
	%%% The next question was about the contribution of the robot to the success of the task. The users had the highest level of agreement for this question which means that the users valued the contribution of the assistance system in task completion. And finally, with respect to the trust that that the user has in the robot, users weakly agreed that they trusted the robot to do the right thing at the right time. 
	%%% \begin{figure}[H]
	%%% \begin{center}
	%%% \includegraphics[width = 0.5\textwidth]{./figures/ModeSwitchCombined}
	%%% \caption{Average number of mode switches for all tasks at all assistance levels - 3D and 2D control interfaces.}
	%%% \label{time}
	%%% \end{center}
	%%% \end{figure}
	%%% All subjects welcomed the customization phase as they were not completely satisfied with the 3 predefined assistance levels. Although, in some cases after a few iterations some of the subjects returned to one of the predefined levels. From the verbal interactions with the subjects, we noticed that reducing task completion time or the number of mode switches were not always the primary motivations for having a preference for a particular assistance level. Users, especially SCI subjects, preferred to retain control as much as possible. 
	%%% Specifically all SCI subjects favored the personal satisfaction that they gained from playing a major role in the task completion than relying on the robot's assistance over better performance. In many cases, the customized assistance levels produced worse performance, in terms of time and number of mode switches. The middle level assistance was kept as the baseline assistance level. The change in the arbitration function from the baseline, characterized by $c_{min}, c_{max}, \alpha_{max}$ during the first customization phase on an average was greater than the change between the second and the first phase (Figure~\ref{AFC}).
	%%% The preferred assistance levels varied across users, although the arbitration functions for different users' were close to the middle level assistance level.
	%
	%%% Figure~\ref{NTC} shows the average number of trials that each category of subjects needed during each customization phase. From the bar graph, it is clear that subjects required more number of trials during the first phase. This can be due to the fact that during the first phase the subjects themselves were getting a feel for how the procedure worked and how to converge to the optimal value slowly.  
	%
	%%% \begin{figure}[H]
	%%% \begin{center}
	%%% \includegraphics[scale = 0.5]{./figures/TimeDiffCatTelCustomA}
	%%% \caption{Task completion times for RfG-Far, across all subject groups for ``teleop'' and ``customA'' assistance levels}
	%%% \label{AFC}
	%%% \end{center}
	%%% \end{figure}
	%
	%%% \begin{figure}[H]
	%%% \begin{center}
	%%% \includegraphics[scale = 0.5]{./figures/MSDiffCatTelCustomA}
	%%% \caption{Number of Mode Switches for RfG-Far, across all subject groups for ``teleop'' and ``customA'' assistance levels}
	%%% \label{AFC}
	%%% \end{center}
	%%% \end{figure}
	%
	%%% \begin{figure}[H]
	%%% %\begin{center}
	%%% \centering
	%%% \includegraphics[width = 0.5\textwidth]{./figures/UserOpinionStdError}
	%%% \caption{The user responses to the questions described in Section IV-D.}
	%%% %\vspace{-1.0em}
	%%% \label{QData}
	%%% %\end{center}
	%%% \end{figure}
	%
	%
	%\section{CONCLUSIONS} \label{conc}
	%This work has introduced a user-driven procedure to customize how
	%control is shared between an assistive robot and human operator.
	%% autonomy with an assistive robotic arm.
	%%Two
	%%tasks with different challenges were designed and three predefined
	%%assistance levels were used for this study. The users were allowed 
	%Our key insight is to parameterize the arbitration function
	%responsible for control sharing, and to allow users to tune the
	%parameter values. A first assessment of the proposed
	%paradigm with real end-users has both affirmed the need for
	%customization and provided guidance for future design of the
	%optimization interface. During the study, subjects were able to tune
	%and optimize the arbitration function in an interactive manner
	%according to their preference. Task performance, in terms of task
	%completion time and the number of mode switches, was considerably
	%better with customized assistance levels compared to pure
	%teleoperation or predefined assistance levels. The user's felt
	%comfortable with the system and trusted the autonomy.
	%%had a high level of trust in the robot
	%%to perform the right action at the right time.  
	%In order to fully understand the effect of task variability on
	%assistance levels, an in-depth longitudinal study, consisting of
	%greater number of tasks with different complexity levels, will be
	%performed in the future. Our on-going work also is currently
	%automating the optimization procedure, with interfaces such as speech
	%recognition and physical slider bars under consideration.
	%%of the system, instead of a system operator a speech
	%%recognition can be used to parse the verbal cues and systematic
	%%changes can be made to the arbitration function.
	%%\textit{Custom A} refers to the customized assistance level, when a task was presented to the subject as the second task and \textit{Custom B} refers to the user preferred assistance level when the same task was presented as the third task. 
	%%The study indicate that the subjects found this procedure
	%%extremely helpful. The t
	%
	%
	%
	%
	%
	%%% \begin{figure}[H]
	%%% \begin{center}
	%%% %\includegraphics[scale = 0.4]{./figures/ArbFuncChangeNumEBNew}
	%%% \includegraphics[scale = 0.4]{./figures/ArbFuncChangeNumEbNew}
	%%% \caption{ Relative change in arbitration function parameters during customization phases.}
	%%% \label{AFC}
	%%% \end{center}
	%%% \end{figure}
	%
	%%% \begin{figure}[H]
	%%% \begin{center}
	%%% \includegraphics[scale = 0.5]{./figures/NoTrialsCustomize}
	%%% \caption{Number of trials during each customization phase.}
	%%% \label{NTC}
	%%% \end{center}
	%%% \end{figure}
	%
	%
	%
	%
	%%A conclusion section is not required. Although a conclusion may review the main points of the paper, do not replicate the abstract as the conclusion. A conclusion might elaborate on the importance of the work or suggest applications and extensions. 
	%%
	% This command serves to balance the column lengths
	%%                                  % on the last page of the document manually. It shortens
	%%                                  % the textheight of the last page by a suitable amount.
	%%                                  % This command does not take effect until the next page
	%%                                  % so it should come on the page before the last. Make
	%%                                  % sure that you do not shorten the textheight too much.
	%%
	%%%%%%%%%%%%%%%%%%%%%%%%%%%%%%%%%%%%%%%%%%%%%%%%%%%%%%%%%%%%%%%%%%%%%%%%%%%%%%%%%%
	%%
	%%
	%%
	%%%%%%%%%%%%%%%%%%%%%%%%%%%%%%%%%%%%%%%%%%%%%%%%%%%%%%%%%%%%%%%%%%%%%%%%%%%%%%%%%%
	%%
	%%
	%%
	%%%%%%%%%%%%%%%%%%%%%%%%%%%%%%%%%%%%%%%%%%%%%%%%%%%%%%%%%%%%%%%%%%%%%%%%%%%%%%%%%%
	%%\section*{APPENDIX}
	%
	%%Appendixes should appear before the acknowledgment.
	%%
	%\section{ACKNOWLEDGMENT}
	%
	%Research reported in this publication was supported by the National
	%Institute of Biomedical Imaging and Bioengineering 
	%%of the National Institutes of Health 
	%under award number R01EB019335. The content is
	%solely the responsibility of the authors and does not necessarily
	%represent the official views of the National Institutes of
	%Health. Many thanks to Jessica Presperin Pedersen, OTR/L, ATP/SMS, for
	%recruiting the SCI subjects, and to Samuel Schlesinger for assistance
	%during the subject studies.
	
	
	
	%
	%%The preferred spelling of the word �acknowledgment� in America is without an �e� after the �g�. Avoid the stilted expression, �One of us (R. B. G.) thanks . . .�  Instead, try �R. B. G. thanks�. Put sponsor acknowledgments in the unnumbered footnote on the first page.
	%
	%
	%
	%%%%%%%%%%%%%%%%%%%%%%%%%%%%%%%%%%%%%%%%%%%%%%%%%%%%%%%%%%%%%%%%%%%%%%%%%%%%%%%%%
	%
	%References are important to the reader; therefore, each citation must be complete and correct. If at all possible, references should be commonly available publications.
	
	
\bibliographystyle{ieeetr}
\bibliography{CASEFinal}






\end{document}
